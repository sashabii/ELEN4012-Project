
\documentclass[10pt,twocolumn]{witseiepaper}
%
% All KJN's macros and goodies (some shameless borrowing from SPL)
\usepackage{KJN}
\usepackage[super]{nth}
\usepackage{subcaption}
\usepackage{listings}
\usepackage{amsmath}
\usepackage{epstopdf}
\usepackage{xcolor}
\usepackage{textcomp}
\usepackage{listings}
\usepackage{alltt}
%\usepackage{matlab-prettifier}
\usepackage{graphicx}
\usepackage{changes}
\usepackage{makecell}
\usepackage{verbatim}
\usepackage{algorithm,algpseudocode}
\usepackage{balance}
\usepackage{pdfpages}
\usepackage{makecell}
\usepackage{color} %red, green, blue, yellow, cyan, magenta, black, white
\definecolor{mygreen}{RGB}{28,172,0} % color values Red, Green, Blue
\definecolor{mylilas}{RGB}{170,55,241}
%\usepackage{flafter}
\usepackage{tikz}
\usetikzlibrary{shapes,arrows}
\usepackage[utf8]{inputenc}
\usepackage[english]{babel}
\usepackage{pgfgantt}
\newcommand\Tstrut{\rule{0pt}{2.2ex}} 
\usepackage{float}

\lstset{language=Matlab, % Set colour for matlab code
	breaklines=true,%
	morekeywords={matlab2tikz},
	keywordstyle=\color{blue},%
	morekeywords=[2]{1}, keywordstyle=[2]{\color{black}},
	identifierstyle=\color{black},%
	stringstyle=\color{mylilas},
	commentstyle=\color{mygreen},%
	showstringspaces=false,%without this there will be a symbol in the places where there is a space
	numbers=left,%
	numberstyle={\tiny \color{black}},% size of the numbers
	numbersep=9pt, % this defines how far the numbers are from the text
	emph=[1]{for,end,break},emphstyle=[1]\color{red}, %some words to emphasise
	%emph=[2]{word1,word2}, emphstyle=[2]{style},    
}
%
% PDF Info
%
\ifpdf
\pdfinfo{
/Title (INSTRUCTIONS AND STYLE GUIDELINES FOR THE PREPARATION OF FINAL YEAR LABORATORY PROJECT PAPERS : 2005 VERSION)
/Author (Ken J Nixon)
/CreationDate (D:200309251200)
/ModDate (D:200510121530)
/Subject (ELEN417/455 Paper Format, 2005)
/Keywords (ELEN417, ELEN455, paper, instructions, style guidelines, laboratory project)
}
\fi

%%%%%%%%%%%%%%%%%%%%%%%%%%%%%%%%%%%%%%%%%%%%%%%%%%%%%%%%%%%%%%%%%%%%%%%%%%%%%%%
\begin{document}


\title{ELEN4012 EMOTION RECOGNITION PROJECT PLAN}

\author{Sasha Berkowitz (818737) \& Arunima Pathania (1117426)
\thanks{School of Electrical \& Information Engineering, University of the
Witwatersrand, Private Bag 3, 2050, Johannesburg, South Africa}
}


%%%%%%%%%%%%%%%%%%%%%%%%%%%%%%%%%%%%%%%%%%%%%%%%%%%%%%%%%%%%%%%%%%%%%%%%%%%%%%%
%
\abstract{ }

\keywords{}

\maketitle
%\thispagestyle{empty}
\pagestyle{plain}
\setcounter{page}{1}


%%%%%%%%%%%%%%%%%%%%%%%%%%%%%%%%%%%%%%%%%%%%%%%%%%%%%%%%%%%%%%%%%%%%%%%%%%%%%%%
\section{INTRODUCTION}

\section{PROBLEM BACKGROUND} % ~1 page
% Artificial intellegence

% How emotional recognition fits in and why it's relevant

\section{PROBLEM SPECIFICATION}\label{spec} % ~1.5 pages
\subsection{Overview}
\subsection{Requirements}
\subsection{Assumptions}
\subsection{Constraints}
\subsection{Success Criteria}


\section{PROPOSED APPROACH}\label{approach} % ~2 pages

\subsection{Approach Overview}
% Flow diagram
% Gantt chart in appendix

\tikzstyle{block} = [draw, fill=white, rectangle, 
minimum height=3em, minimum width=3em]
\tikzstyle{sum} = [draw, fill=white, circle, node distance=1cm]
\tikzstyle{input} = [coordinate]
\tikzstyle{output} = [coordinate]
\tikzstyle{pinstyle} = [pin edge={to-,thin,black}]

\begin{figure*}[h]
	\centering
	\begin{tikzpicture}[auto, node distance=2.5cm,>=latex']
	
	\node [input, name=input] {};
	
	\node [block, right of=input,text width=1.7cm, text centered] (sensor) {Noise Cancellation};
	\node [block, right of=sensor,node distance=2.5cm, text width=1.7cm, text centered] (cond1) {Signal Enhancement};
	\node [block, right of=cond1,node distance=3cm, text width=2.5cm, text centered] (cond2) {Classification};
	%\node [block, right of=cond2, node distance=2.5cm, text width=2.05cm, text centered] (cond3) {Voltage Comparator};
	%\node [block, right of=cond3,node distance=2.5cm, text width=1.7cm, text centered] (proc) {Micro-Controller};
	%\node [block, right of=proc,node distance=2.5cm, text width=1.7cm, text centered] (disp) {Display};
	%	\node [block, right of=controller,node distance=2.5cm] (system) {System};
	
	\draw [->] (sensor) -- node[name=u] {$ $} (cond1);
	\draw [->] (cond1) -- node[name=v] {$ $} (cond2);
	%\draw [->] (cond2) -- node[name=v] {$ $} (cond3);
	%\draw [->] (cond3) -- node[name=v] {$ $} (proc);
	%\draw [->] (proc) -- node[name=v] {$ $} (disp);
	\node [output, right of=cond2] (output) {};
	\coordinate [below of=u] (measurements) {};
	
	\draw [draw,->] (input) -- node {$input$} (sensor);
	\draw [->] (cond2) -- node [name=y] {$ output$}(output);
	
	\end{tikzpicture}
	\caption{Block diagram of approach overview.}
	\label{fig:sysOverview}
\end{figure*}

\subsection{Approach Details}
\subsubsection{Speech Input}
Speech will be sourced from recordings taken of students. These will be done in a soundproof chamber in order to reduce external interruption. In order ot reduce costs they will be done one of the student's iPhones which record at as M4A files, which will be converted to .WAV files for use. *Recording specs* [reference paper]

\subsubsection{Signal Enhancement}
% ARU
% Please discuss the signal conditioning elements here and how they work (Look at his example report he gave us for layout)
% 1 -> NPVSS-NLMS
% 2 -> GSVSS-NLMS
% 3&4 - Please find two other suited algorithms
checking
\subsubsection{Signal Classification}
% Tensorflow

% Neural net model

% Number of hidden layers chosen

\subsubsection{Output}
% Outputed at the interface as the emotion 

\subsubsection{User Interface}
% UI developed in Python
% Users will be able to record inputs to be sent through the network

\subsection{Training \& Testing}
% Describe training, validatin, testing works in neural net

% Data gathered from students and from databases online

% Ratios used

\section{PROJECT MANAGEMENT} % ~0.5 pages 
Throughout the duration of the project, the partners will be in contact with each other regarding their progress or any other obstacles which they may face. In addition, at least one weekly face-to-face meeting will be held between the partners and project supervisor.

Project files will be housed on a private GitHub repository.

The subsections following will outline how the project components listed in section \ref{approach} will be divided and the estimated time to be taken for each.

\subsection{Prerequisites}
Ethics Clearance

Installing all needed for project

Set up project repo

\textbf{Resources required:} Git

\subsection{Data Collection}

Source training data

Meet with students to record data

\textbf{Resources required:} Microphone, Recording application, access to soundproof chamber


\subsection{Signal Enhancement}

\textbf{Resources required:} Python 3.5


\subsection{Signal Classification}

\textbf{Resources required:} Python 3.5, TensorFlow


\subsection{User Interface}

\textbf{Resources required:} Python, TKinter


\section{RISKS \& THEIR MITIGATIONS} % ~0.3 pages
\subsection{Data Security}
Data Stolen and misused.

Will be mitigated by storing all collected data in password protected file and destroyed after.

\subsection{Ethical Issues}
Personal information disclosed, etc.

Ethics clearance will be obtained prior to project. A script will be given to participants therefore no personal info. Each participant will be given information sheet on the project and sign a declaration.

\subsection{Inaccurate Results}
Data could be distorted and give inaccurate results.

Recorded in soundproof chamber where possible. Tests will be run on data. Estimated percentage accuracy will be obtained.

\subsection{Intellectual Property Risks}
Google
%%%%%%%%%%%%%%%%%%%%%%%%%%%%%%%%%%%%%%%%%%%%%%%%%%%%%%%%%%%%%%%%%%%%%%%%%%%%%%%
\section{CONCLUSION} % ~0.2 pages


%%%%%%%%%%%%%%%%%%%%%%%%%%%%%%%%%%%%%%%%%%%%%%%%%%%%%%%%%%%%%%%%%%%%%%%%%%%%%%%
%
%\nocite{*}
\bibliographystyle{witseie}
\bibliography{prelim}

\newpage
\onecolumn

\begin{appendix}
	
	\section{Gantt Chart}
	
		\begin{ganttchart}{1}{12}
			\gantttitle{2011}{12} \\
			\gantttitlelist{1,...,12}{1} \\
			\ganttgroup{Group 1}{1}{7} \\
			\ganttbar{Task 1}{1}{2} \\
			\ganttlinkedbar{Task 2}{3}{7} \ganttnewline
			\ganttmilestone{Milestone}{7} \ganttnewline
			\ganttbar{Final Task}{8}{12}
			\ganttlink{elem2}{elem3}
			\ganttlink{elem3}{elem4}
		\end{ganttchart}
	
	
	
\end{appendix} 	

{\tiny \vfill \hfill \today \hspace{5mm} witseie-paper-2003.\TeX}


\end{document}

" vim: ts=4
" vim: tw=78
" vim: autoindent
" vim: shiftwidth=4
